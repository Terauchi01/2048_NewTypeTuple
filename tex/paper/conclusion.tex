\section{Conclusion}
% 本研究では,2048 における強力なNタプルネットワークの設計に関する課題に取り組んだ。
% Nタプルネットワークは一般に,タプルのサイズを大きくするほど表現力が増し,性能の向上が期待できる。
% しかし同時に,タプルのサイズを大きくするたびにパラメータ数が指数的に増加し,必要なメモリと学習時間が膨大になるという深刻な問題を抱えていた。
% 例えば,6タプルでは数千万のパラメータで済むが,7タプルでは数億,8タプルでは数十億のパラメータが必要となり,従来の手法では実用的に扱うことが不可能であった。
% この問題を解決するために,本研究では新しい符号化手法である Vertical Split Encoding (VSE) を提案した。
% まず Mini2048 を用いた予備実験により,VSE が性能を大きく損なうことなく学習に利用できることを確認した。
% さらに 6タプル,7タプル(2-VSE),8タプル(3-VSE),9タプル(4-VSE)のネットワークを用いて2048のプレイヤを構築し,学習と評価を行った。
% その結果,VSE を適用した大規模タプルネットワークはいずれも6タプルのベースラインを上回った。
% 特に 8タプル+3-VSE は,5-ply Expectimax 探索で 594,619 という高いスコアを達成した。これは NT6-M との差が約70,000点に及び,標準偏差を十分に上回る有意な差であった。
% これにより,VSE が大規模 Nタプルネットワークの性能を引き出す有効な手法であることを示した。
% 本研究の主な貢献は三点にまとめられる。第一に,従来はメモリや学習時間の制約から現実的に扱うことが困難であった大規模タプルを,VSE を用いることで実用的に利用可能にした点である。
% 第二に,提案手法を用いた大規模Nタプルネットワークが,従来最先端であった6タプルベースのプレイヤを上回る性能を示すことを実験的に確認した点である。
% 第三に,Nタプルネットワーク研究において「表現力を維持しつつパラメータを効率化する」という新しい方向性を提示した点である。
% 今後の課題として,まず VSE における値帯域の分割設計は今回手作業および経験則に基づいて設定した。
% しかし,分割方法によって性能やパラメータ数が大きく変化するため,自動化や最適化が求められる。
% また,大規模タプルネットワーク,特に NT8 以上では学習の安定性に課題が残っており,学習率制御や正則化,初期化戦略の改良といった手法の導入が有効と考えられる。
% これらの課題に取り組むことで,2048 におけるさらなる性能向上や,他のゲーム・領域への応用が期待される。
