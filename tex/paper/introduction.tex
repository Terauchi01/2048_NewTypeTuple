\section{Introduction}

N-tuple networks (a.k.a. pattern-based evaluation functions in Othello~\cite{Buro98}) are a simple but
efficient approach to designing evaluation functions for board games and other applications.
N-tuple networks compute an evaluation value by sampling local features on the board, retrieving the corresponding values from the lookup tables, and summing them up (or calculating a linear combination of them).
The parameters in the lookup tables are often adjusted by supervised learning or reinforcement learning techniques.
Several strong computer players have been developed based on N-tuple networks, for example,  for Othello~\cite{Buro98,Jask14,NgIL12,KuSM25}, \mbox{connect-4}~\cite{ThKK12}, connect-6~\cite{HuQP18}, EinStein W\"urfelt Nicht!~\cite{CHHH24,HsHs25}.

The target game in this study is \emph{2048}~\cite{2048}, a single-player stochastic game developed by G. Cirulli.
Several computer players have been implemented \cite{Mats21,ASOH22,GuCW22,SzJa14,YWHC16,Jask18,KGWW22,Zhou19,WaMa25}, among which the most successful approaches employ N-tuple networks as evaluation functions~\cite{SzJa14} in combination with Expectimax search~\cite{YWHC16}.
For instance, the state-of-the-art player developed by Guei et al.~\cite{GuCW22} achieved an average score of 625\,377 using two-stage N-tuple networks consisting of 8 $\times$ 6-tuples---trained by extended temporal-difference learning with optimistic initialization---, 6-ply Expectimax search, and a game-specific tile-downgrading technique.


In general, larger N-tuple networks yield better performance at the cost of larger memory sizes required and longer training times for tuning~\cite{TeMa25}.
Let N-tuple networks consist of $m$ $\times$ $n$-tuples, then we can consider two approaches to enlarging N-tuple networks.
The first approach is to increase the number $m$ of tuples.  Although this is relatively straightforward to implement, Oka and Matsuzaki~\cite{OkMa16} demonstrated that performance was saturated only with this approach.
The second approach is to increase the size $n$ of the tuples.  In the early stages of research on 2048, computer players improved performance by enlarging 4-tuples to 6-tuples~\cite{SzJa14,YWHC16}.  However, as Ja\'skowski~\cite{Jask18} suggested, further enlargement has been considered infeasible for game 2048 due to the exponential growth in the number of parameters.  Note that, in 2048, a board cell can take 18 possible values, and N-tuple networks with $m$\,$\times$\,$n$-tuples have $m\times 18^n$ parameters.  A 6-tuple network has $3.4\times 10^7$ parameters (requiring 13 GB of memory~\footnote{The memory size is calculated for the case that 8\,$\times$\,$n$-tuple networks with 2-stage, 64 bits per parameter, are tuned with temporal coherence learning~\cite{Jask18}.}), a 7-tuple network has $6.1\times 10^8$ parameters (235 GB), and an 8-tuple has $1.1\times 10^{10}$ parameters (4.2 TB).


In this study, we overcome the aforementioned limitation by proposing a novel encoding method, called \emph{Vertical Split Encoding} (\emph{VSE}). In $k$-VSE, we encode a board state into $k$ board instances based on the $k$ value ranges of interest. Here, for each value range, the encoded board instance takes fewer possible values by equating values outside the range (except for empty cells, represented as $E$).
For example, with 2-VSE with value ranges $2^1$--$2^8$ and $2^9$--$2^{17}$, a board $[E, 2^1, 2^{11}, 2^{12}, \ldots]$ is encoded in $[E, 2^1, L, L]$ and $[E, S, 2^{11}, 2^{12}]$, where $L$ and $S$ are newly introduced special labels denoting \emph{larger} and \emph{smaller} values, respectively.
With VSE, the number of parameters required for $m$\,$\times$\,$n$-tuples in 2048 is reduced to $m\times 11^n$ with 2-VSE, $m\times 9^n$ with 3-VSE, and $m\times 7^n$ with 4-VSE. The reduction is drastic: for example when $n=8$, the numbers of parameters are reduced to 3.9\% with 2-VSE, 1.2\% with 3-VSE, and 0.21\% with 4-VSE, respectively.


We demonstrate the effectiveness of VSE for 2048 with intensive experiments.
In addition to 6-tuple networks without VSE, we developed computer players using larger N-tuple networks: 7-tuple networks with 2-VSE, 8-tuple networks with 3-VSE, and even 9-tuple networks with 4-VSE.
All N-tuple networks are successfully trained on a commodity computer using at most 64 GB of memory, and the 7-, 8- and 9-tuple networks outperformed the 6-tuple baseline.
In particular, the best 8-tuple networks achieved an average score of 407\,206 with 1-ply lookahead (greedy) play, 547\,365 with 3-ply Expectimax search, and 587\,690 with 5-ply Expectimax search, which were significantly better than the 6-tuple baseline.

% 本研究の主な貢献は以下の2点である。
% 第一に,従来はメモリや学習時間の制約から現実的に扱うことが困難であった大規模タプルを実現可能とする,VSE を提案したことである.
% 第二に,VSEを用いた大規模Nタプルネットワークが,最先端プレイヤで用いられている6タプルベースのプレイヤを同一条件にて性能で上回ることを実験的に確認した点である。

% 本研究は,様々なゲームにおいて用いられているNタプルネットワークにおいてパラメータを削減する新しい方法を提案するものである.

The main contributions of this study are twofold.
\begin{itemize}
 \item We proposed Vertical Split Encoding (VSE) to enable the employing of large $N$-tuple networks, which had been impractical due to memory constraints.
 \item We experimentally demonstrated that large N-tuple networks with VSE outperformed the 6-tuple baseline in the same experimental setting.
\end{itemize}
More broadly, this work introduces a novel approach for reducing parameters in N-tuple networks, which would be widely applied across various games.

% LocalWords:  EinStein urfelt Nicht Cirulli Guei et al expectimax Oka
% LocalWords:  Ja skowski VSE lookahead
